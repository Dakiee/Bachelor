\begin{abstract}
	Орчин үеийн ертөнц дижитал харилцаа холбоог хөгжүүлж, шивэх хурд, нарийвчлал нь янз бүрийн мэргэжлээр ажилладаг хүмүүст зайлшгүй шаардлагатай ур чадвар болж байна. "Duck Racer" нь шивэх чадвараа сайжруулах, сурах үйл явцыг тоглоом болгох зорилготой вебэд суурилсан программ юм. Энэхүү систем нь өрсөлдөөнт уралдааны механизмыг нэгтгэсэн бөгөөд оролцогчид өгөгдсөн догол мөрүүдийг бичиж бодит цагт тоглогчдын эсрэг уралдана. Хэрэглэгчдэд сонирхолтой интерфэйс, бодит цагийн гүйцэтгэлийн хэмжигдэхүүн, текстийн хэсгээс бүрдсэн олон төрлийн мэдээллийг цогцоор нь хослуулж, шивэх урлагийг эзэмших өвөрмөц аргыг хэрэглэгчдэд санал болгож байна.

	\setcounter{secnumdepth}{0}

	\section{Зорилго}
	Шивэх дасгалыг үр дүнтэй, тааламжтай болгох системийг тоглоомын аргаар сургалтын үйл явцад нэгтгэж шивэх дасгалыг сонирхолтой, хэрэглэгчдийг өрсөлдөх чадвартай болгохыг зорьж байна.

	\section{Зорилт}
	Уг веб аппыг хөгжүүлэхдээ дараах үе шатын дагуу ажиллана.
	\begin{enumerate}
		\item Техникийн болон хэрэглэгчийн шаардлагыг тодорхойлох;
		\item Хэрэглэгчдийн анхаарлыг татахуйц хэрэглэгчдэд ээлтэй интерфэйсийг дизайн гаргах;
		\item Ашиглах технологийг онол болон практик дээр суурилж судлах;
		\item Системийн архитектурын бүтэц, дизайныг зохион байгуулж бэлдэх;
		\item Гаргасан баримт бичгийн дагуу системийн хөгжүүлэлтээ хийх;
		\item Бэлэн болсон системд домейн нэр авж, хост хийн байршуулах.
	\end{enumerate}

	\section{Сэдэв сонгох үндэслэл}

	Энэхүү дипломын ажил нь шивэх чадварыг сайжруулах өвөрмөц боловсролын систем бөгөөд. Энэ сэдвийг сонгосон нь хэд хэдэн чухал шалтгаанаас үүдэлтэй:

	\begin{enumerate}
		\item Монголын хүн амын гуравны нэгээс илүү хувь нь 24-өөс доош насныхан байдаг бөгөөд, тоглоомын платформоор залуучуудыг татан оролцуулах нь тэднийг дижитал эринд илүү сайн бэлтгэж чадна.
		\item COVID-19 тахал гэх мэт нөхцөл байдлаас үүдэн хурдассан онлайн боловсрол руу дэлхий даяар шилжиж байгаа нь шивэх чадвар сайтай байх шаардлагатайг улам тодотгосон.\footnote{Боловсролын талаарх ЮНЕСКО-гийн тайлан: \url{https://www.unesco.org/en/education}}
		\item  Орчин үеийн ажлын байрууд салбараас үл хамааран харилцаа холбоо, хамтын ажиллагаа, бичиг баримт бүрдүүлэхэд дижитал хэрэгслүүд ихээхэн ашигладаг. шивэх ур чадвар нь бүтээмжийг дээшлүүлж, алдааг багасгаж, ажлын урсгалыг илүү хялбар болгоход хувь нэмэр оруулна.\footnote{LinkedIn ажлын зах зээлийн албан ёсны тайлан: \url{https://economicgraph.linkedin.com/resources}}
		\item Вэб дээр суурилсан платформ нь дэлхий даяарх хэрэглэгчдийг холбох боломжтой. Энэхүү дэлхийн холболт нь соёлын солилцоог дэмжиж, эрүүл өрсөлдөөнийг дэмжиж, суралцах нийгэмлэгийг бий болгож чадна.\footnote{Pew судалгааны төв: \url{https://www.pewresearch.org/}}
	\end{enumerate}

	\section{Ач холбогдол}
	Уг системийг бүтээснээр тоглоомын сорилтуудаар дамжуулан хэрэглэгчид шивэх хурд, нарийвчлалыг цаг хугацааны явцад сайжруулахад түлхэц болох.
	Сургууль, боловсролын байгууллагууд хичээлд нэмэлт хэрэгсэл болгон ашиглах. Оюутнууд дадлага хийж, ахиц дэвшлээ хянаж, ангийнхантайгаа хөгжилтэй, интерактив байдлаар өрсөлдөж, сургалтын үйл явцыг илүү сонирхолтой байх боломжуудыг бүрдүүлэх юм.
	\setcounter{secnumdepth}{2}

\end{abstract}
