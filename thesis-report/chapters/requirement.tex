\chapter{Системийн шаардлага}

Уг бүлэг нь системийн хэрэглэгчийн зүгээс тавигдах шаардлагыг тодорхойлж, хэрэглэгч талаас ээлтэй хэрэглэхэд амархан байх тал дээр UX болон UI дизайны
шаардлагуудыг гаргасан.

\section{Шаардлагын шинжилгээ}

\subsection{Хэрэглэгчид}

Интернэт сүлжээ ашиглан хурдан шивэх хүсэлтэй бүх төрлийн хэрэглэгчид хэрэглэх боломжтой.

\subsection{Функционал шаардлагууд}

\begin{table}[h]
	\centering
	\caption{Функциональ шаардлага}
	\begin{tabular}{ |p{2cm}|p{13cm}| }
		\hline
		ФШ 101 & Бодит цагт мэдээллүүдийг харуулах (хурд, нарийвчлал, одоогийн байрлал).           \\ \hline
		ФШ 102 & Алдсан, буруу бичсэн үг эсвэл тэмдэгтүүдийг тодруулах.                            \\ \hline
		ФШ 103 & Хэрэглэгчид өгөгдсөн текстнээс хамаарч дуусгахад шаардагдах хугацааг өөрчлөх.     \\ \hline
		ФШ 104 & Хэрэглэгчид unique холбоос ашиглан тодорхой найзуудтайгаа өрсөлдөх боломжтой байх. \\ \hline
		ФШ 105 & Уралдааны үеэр бүх оролцогчдын бодит цагийн явцыг харуулах.                        \\ \hline
		ФШ 106 & шивэх хурд, нарийвчлал дээр үндэслэн тэргүүлэгчдийн самбарыг харуулах.             \\  \hline
		ФШ 107 & Веб нь хэрэглэгч бүртгэх боломжтой байх.                                           \\ \hline
	\end{tabular}
\end{table}

\subsection{Функционал бус шаардлагууд}

\begin{table}[h]
	\centering
	\caption{Функциональ бус шаардлага}
	\begin{tabular}{ |p{2cm}|p{13cm}| }
		\hline
		ФБШ 101 & Уралдааны үеэр бодит цагийн хариу үйлдэл үзүүлэг байх.                                  \\ \hline
		ФБШ 103 & Систем нь дор хаяж 2 хэрэглэгчид хоорондоо өрсөлддөг байх.                               \\ \hline
		ФБШ 104 & SQL injection, CSRF халдлагууд болон бусад нийтлэг вэб эмзэг байдлаас хамгаалттай байх. \\ \hline
		ФБШ 105 & Хэрэглэгчийн өгөгдөл болон тоглоомын статистикийг тогтмол нөөцлөдөг байх.               \\ \hline
		ФБШ 102 & Интерфейс, UI нь шинэ хэрэглэгчдэд зориулсан ойлгомжтой байх.                            \\ \hline
		ФБШ 106 & Төрөл бүрийн хөтчүүдэд нийцтэй байх.      \\  \hline
	\end{tabular}
\end{table}

\section{UX/UI шаардлага}

UI/UX (User Interface/User Experience) шаардлагууд нь програм хангамж, вэбсайт эсвэл аливаа дижитал бүтээгдэхүүн боловсруулахад хэд хэдэн шалтгааны улмаас зайлшгүй шаардлагатай байдаг.

\begin{enumerate}
	\item Хэрэглэгч төвтэй дизайн -> Хэрэглэгчийн хэрэгцээ, сонголтыг ойлгож, баримтжуулснаар хэрэглэгчийн эерэг туршлагыг бий болгох боломжтой.
	\item Эрсдэлийг бууруулана -> Шаардлагууд нь хөгжлийн дараагийн үе шатанд хамрах хүрээ болон өөрчлөлтөөс урьдчилан сэргийлэхэд тусална.
	\item Зохион байгуулалт, өнгө, хэв маяг, загвар зэрэг дизайны элементүүдийн дагнасан байдал нь хэрэглэгчдэд системийг удирдах, ойлгоход хялбар болгодог.
\end{enumerate}

Дизайны тодорхой шаардлагыг эхнээс нь хэрэгжүүлэх нь хэрэглэгчийн туршлагад сөргөөр нөлөөлөхөөс өмнө ашиглалтын асуудлыг шийдвэрлэхэд тусалдаг давуу талтай.

\subsection{User Experience шаардлага}

UX шаардлагууд нь хэрэглэгчийн хэрэгцээ, сонголт, хүлээлтийг ойлгох, хангахад зайлшгүй шаардлагатай.

\begin{table}[h]
	\centering
	\caption{User Experience дизайны шаардлага}
	\begin{tabular}{ |p{2cm}|p{13cm}| }
		\hline
		ИДШ 101 & Зорилтот хэрэглэгчдийн зорилгыг ойлгохын тулд хэрэглэгчийн судалгаа хийх.                                                 \\ \hline
		ИДШ 102 & Цэсийг хөнгөвчлөхийн тулд агуулга, онцлогуудыг логик, уялдаатай байдлаар зохион байгуулах.                 \\ \hline
		ИДШ 103 & Янз бүрийн төхөөрөмж болон дэлгэцийн хэмжээтэй сайн ажилладаг интерфэйсийг зохион бүтээх.                       \\ \hline
		ИДШ 104 & Брэнд, хэрэглэгчийн хүлээлттэй нийцсэн сэтгэл татам, үзэмжтэй интерфэйсийг зохион бүтээх.                         \\ \hline
	\end{tabular}
\end{table}

\subsection{User Interface дизайны шаардлага}

UI шаардлагууд нь үр дүнтэй, харагдахуйц хэрэглэгчийн интерфэйсийг бий болгоход зайлшгүй шаардлагатай.

\begin{table}[h]
	\centering
	\caption{User Interface дизайны шаардлага}
	\begin{tabular}{ |p{2cm}|p{13cm}| }
		\hline
		ИДШ 105 & Хэрэглэгчдийн анхаарлыг чухал элементүүд болон контентод чиглүүлэхийн тулд тодорхой харааны шатлалыг бий болгох.                                                 \\ \hline
		ИДШ 106 & Гол элементүүд дээр "3A3845" hex кодтой өнгийг ашиглах.                 \\ \hline
		ИДШ 107 & Текст элементүүдийн үсгийн хэлбэр, үсгийн хэмжээ, мөр хоорондын зай, шатлал зэргийг зааж өгөх.                       \\ \hline
		ИДШ 108 & Цэс, навигацийн мөр, site map зэрэг навигацийн бүтцийг тодорхойлох.                         \\ \hline
		ИДШ 109 & Contrast буюу өнгөний ялгарлыг бага байлгах.                            \\ \hline
		ИДШ 110 & Элемент хооронд white-space буюу сул зайг сайн гаргаж өгөх.             \\  \hline
	\end{tabular}
\end{table}