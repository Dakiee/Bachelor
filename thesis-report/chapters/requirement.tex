\chapter{Системийн шаардлага}

Уг бүлэг нь системийн хэрэглэгчийн зүгээс тавигдах шаардлагыг тодорхойлж, тухайн гаргасан шаардлагууд дээрээ үндэслэн UX судалгаа хийсэн талаарх гарах ба хэрэглэгч суурьтай интерфейс дизайн гаргахад тулгарсан асуудлуудыг товч дурдлаа.

\section{Шаардлагын шинжилгээ}

\subsection{Хэрэглэгчид}

Интернет сүлжээ ашиглан мэдээлэл авдаг, бусадтай хуваалцдаг бүх төрлийн хэрэглэгчид 

\subsection{Функционал шаардлагууд}

\begin{table}[h]
	\centering
	\caption{Функциональ шаардлага}
	\begin{tabular}{ |p{2cm}|p{13cm}| }
		\hline
		ФШ 101 & Бодит цагт мэдээллүүдийг харуулах (хурд, нарийвчлал, одоогийн байрлал).           \\ \hline
		ФШ 102 & Алдсан, буруу бичсэн үг эсвэл тэмдэгтүүдийг тодруулах.                            \\ \hline
		ФШ 103 & Хэрэглэгчид өгөгдсөн текстнээс хамаарч дуусгахад шаардагдах хугацааг өөрчлөх.     \\ \hline
		ФШ 104 & Хэрэглэгчид unique холбоос ашиглан тодорхой найзуудтайгаа өрсөлдөх боломжтой байх \\ \hline
		ФШ 105 & Уралдааны үеэр бүх оролцогчдын бодит цагийн явцыг харуулах                        \\ \hline
		ФШ 106 & Бичих хурд, нарийвчлал дээр үндэслэн тэргүүлэгчдийн самбарыг харуулах             \\  \hline
		ФШ 107 & Веб нь хэрэглэгч бүртгэх боломжтой байх                                           \\ \hline
	\end{tabular}
\end{table}

\subsection{Функционал бус шаардлагууд}

\begin{table}[h]
	\centering
	\caption{Функциональ бус шаардлага}
	\begin{tabular}{ |p{2cm}|p{13cm}| }
		\hline
		ФБШ 101 & Уралдааны үеэр бодит цагийн хариу үйлдэл үзүүлэг байх.                                  \\ \hline
		ФБШ 103 & Систем нь дор хаяж 2 хэрэглэгчид хоорондоо өрсөлддөг байх                               \\ \hline
		ФБШ 104 & SQL injection, CSRF халдлагууд болон бусад нийтлэг вэб эмзэг байдлаас хамгаалттай байх. \\ \hline
		ФБШ 105 & Хэрэглэгчийн өгөгдөл болон тоглоомын статистикийг тогтмол нөөцлөдөг байх.               \\ \hline
		ФБШ 102 & Интерфейс UI нь шинэ хэрэглэгчдэд зориулсан ойлгомжтой байх.                            \\ \hline
		ФБШ 106 & Төрөл бүрийн төхөөрөмж (компьютер, таблет, гар утас) болон хөтчүүдэд нийцтэй байх.      \\  \hline
	\end{tabular}
\end{table}

\section{UX/UI шаардлага}

Уг судалгааны ажлын онцлог тал нь шууд хөгжүүлэлтээ хийж эхлэхээс өмнө хэрэглэгч төвтэй дизайн гаргаж түүнийгээ зорилтот хэрэглэгч дээр туршиж, гүйцэтгэл сайтай User Experience болон User Interface дизайн гаргах юм. Ингэснээр сайн бүтээгдэхүүн гаргах том үндэс болох, ирээдүйд гарах хөгжүүлэлтийн зардлыг багасгах давуу талтай.

\subsection{User Experience шаардлага}

Хэрэглэгч төвтэй UX/UI дизайн гаргахад дараах үе шатын дагуу ажиллах шаардлагатай.

\textbf{- User Persona гаргах}

Манай платформыг ашиглах боломжтой хоёроос дээш хэрэглэгчийг сонгож урьдчилж бэлдсэн асуултуудаас асууж мэдээллийг нэгтгэн тухайн хүнийг тодорхой хэмжээнд дүгнэн, уг платформыг ямар зорилготойгоор ашиглах боломжтой нөхцлүүдийг /Use Case/ гаргана.

\textbf{- Асуудал тодорхойлох}

Гаргасан Use Case дээрээ үндэслэж эцсийн хэрэглэгч дээр ямар асуудлууд байгааг судалж, хэрхэн шийдэх талаар санаа гарган User Experience-н шаардлагуудыг тодорхойлно. Энэ нь хэрэглэгч төвтэй дизайн гаргахын үндэс болох тул таамгаар бус судалгаан дээрээ үндэслэж вэбийн хэрэглэгчийн харилцах хэсгийн дизайныг гүйцэтгэнэ.

\textbf{- Доод түвшинд Prototype хувилбар гаргах}

Дээрх ажлуудыг нэгтгэн User Experience дизайныг доод түвшинд буюу ерөнхий загвартайгаар хийж, ашиглаж буй хэрэгсэл болох Figma дээрээ бүх хуудас, компонентийн логик үйлдлүүдийг холбож бодит ажилладаг мэт загвар гаргана.

\textbf{- Usability туршилт хийж дизайнаа сайжруулах}

Гаргасан Prototype хувилбараа сонгож авсан хэрэглэгчдээр ашиглиулж UX дээр ямар алдаа байгаа, хэрэглэгчдийн асуудлыг шийдвэрлэж чадаж байгаа эсэх, вэб компонентүүдийн байрлал, хоорондын логик үйлдэл зөв байгаа эсэхийг тодорхойлж хэрэглэгчдээс санал хүсэлт авах шаардлагатай. Түүний дараагаас дизайнаа дахин сайжруулж, хөгжүүлэлтийн шатанд ороход бэлэн болгох хэрэгтэй.

\subsection{User Interface дизайны шаардлага}

\begin{table}[h]
	\centering
	\caption{User Interface дизайны шаардлага}
	\begin{tabular}{ |p{2cm}|p{13cm}| }
		\hline
		ИДШ 101 & Шинэлэг дизайнтай байх                                                 \\ \hline
		ИДШ 102 & Гол элементүүд дээр "D85888" hex кодтой өнгийг ашиглах                 \\ \hline
		ИДШ 103 & Компонентуудын micro-interaction ойлгомжтой байх                       \\ \hline
		ИДШ 104 & Дэвсгэр өнгө дээр цагаан өнгийг түлхүү ашиглах                         \\ \hline
		ИДШ 105 & Contrast буюу өнгөний ялгарлыг бага байлгах                            \\ \hline
		ИДШ 106 & Веб фонтын хувьд "Rubik - Cyrillic Extented" хувилбарыг ашигласан байх \\ \hline
		ИДШ 107 & Элемент хооронд white-space буюу сул зайг сайн гаргаж өгөх             \\  \hline
	\end{tabular}
\end{table}