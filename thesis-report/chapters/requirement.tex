\chapter{Системийн шаардлага}

Уг бүлэг нь системийн хэрэглэгчийн зүгээс тавигдах шаардлагыг тодорхойлж, тухайн гаргасан шаардлагууд дээрээ үндэслэн UX судалгаа хийсэн талаарх гарах ба хэрэглэгч суурьтай интерфейс дизайн гаргахад тулгарсан асуудлуудыг товч дурдлаа.

\section{Шаардлагын шинжилгээ}

\subsection{Хэрэглэгчид}

Интернет сүлжээ ашиглан мэдээлэл авдаг, бусадтай хуваалцдаг бүх төрлийн хэрэглэгчид

\subsection{Функционал шаардлагууд}

\begin{table}[h]
	\centering
	\caption{Функциональ шаардлага}
	\begin{tabular}{ |p{2cm}|p{13cm}| }
	\hline
	ФШ 101 &  Веб нь бусад веб холбоосуудыг дангаар нь болон бүлэглэж оруулах боломжтой байх \\ \hline
	ФШ 102 &  Нийт хэрэглэгчдийн оруулсан веб холбоосууд, хэрэглэгчдийн мэдээллээс түлхүүр үгээр хайлт хийх боломжтой байх \\ \hline
	ФШ 103 &  Хэрэглэгч бүлэглэж оруулсан холбоосуудаа бусад хүмүүстэй хуваалцах боломжтой байх \\ \hline
	ФШ 104 &  Хэрэглэгч платформ дээрх дурын хүнээ дагах, түүний оруулсан веб холбоосуудыг харах боломжтой байх \\ \hline
	ФШ 105 &  Бусад хүмүүсийн оруулсан веб холбоосууд дээр хэрэглэгч үнэлгээ өгдөг боломжтой байх \\ \hline
	ФШ 106 &  Веб нь хэрэглэгчийн дагасан сэдвийн дагуу холбоосуудыг харуулдаг байх \\  \hline
	ФШ 107 &  Веб нь хэрэглэгч бүртгэх боломжтой байх \\ \hline
\end{tabular}
\end{table}

\subsection{Функционал бус шаардлагууд}

	\begin{table}[h]
		\centering
		\caption{Функциональ бус шаардлага}
		\begin{tabular}{ |p{2cm}|p{13cm}| }
		\hline
		ФБШ 101 &  Веб нь хэрэглэгч ашиглахад хялбар интерфейстэй байх \\ \hline
		ФБШ 102 &  Интерфейс дизайн нь UI шаардлагын дагуу хөгжүүлэгдсэн байх \\ \hline
		ФБШ 103 &  Веб дээр нийтлэл оруулах үед бусад веб холбоосуудын Open Graph мэдээллийг 500 миллсекундэд багтаан авдаг байх \\ \hline
		ФБШ 104 &  Вебийн эхний хувилбар зөвхөн Desktop төхөөрөмж дээр ажилладаг байх \\ \hline
		ФБШ 105 &  Веб холбоос оруулах үед дээд тал нь хоёр hashtag ашигладаг байх \\ \hline
		ФБШ 106 &  Хэрэглэгч хэдэн ч удаа веб холбоос оруулах боломжтой байх \\  \hline
	\end{tabular}
	\end{table}

\section{UX/UI шаардлага}

Уг судалгааны ажлын онцлог тал нь шууд хөгжүүлэлтээ хийж эхлэхээс өмнө хэрэглэгч төвтэй дизайн гаргаж түүнийгээ зорилтот хэрэглэгч дээр туршиж, гүйцэтгэл сайтай User Experience болон User Interface дизайн гаргах юм. Ингэснээр сайн бүтээгдэхүүн гаргах том үндэс болох, ирээдүйд гарах хөгжүүлэлтийн зардлыг багасгах давуу талтай. 

\subsection{User Experience шаардлага}

Хэрэглэгч төвтэй UX/UI дизайн гаргахад дараах үе шатын дагуу ажиллах шаардлагатай.

\textbf{- User Persona гаргах} 

Манай платформыг ашиглах боломжтой хоёроос дээш хэрэглэгчийг сонгож урьдчилж бэлдсэн асуултуудаас асууж мэдээллийг нэгтгэн тухайн хүнийг тодорхой хэмжээнд дүгнэн, уг платформыг ямар зорилготойгоор ашиглах боломжтой нөхцлүүдийг /Use Case/ гаргана.

\textbf{- Асуудал тодорхойлох}

Гаргасан Use Case дээрээ үндэслэж эцсийн хэрэглэгч дээр ямар асуудлууд байгааг судалж, хэрхэн шийдэх талаар санаа гарган User Experience-н шаардлагуудыг тодорхойлно. Энэ нь хэрэглэгч төвтэй дизайн гаргахын үндэс болох тул таамгаар бус судалгаан дээрээ үндэслэж вэбийн хэрэглэгчийн харилцах хэсгийн дизайныг гүйцэтгэнэ.

\textbf{- Доод түвшинд Prototype хувилбар гаргах}

Дээрх ажлуудыг нэгтгэн User Experience дизайныг доод түвшинд буюу ерөнхий загвартайгаар хийж, ашиглаж буй хэрэгсэл болох Figma дээрээ бүх хуудас, компонентийн логик үйлдлүүдийг холбож бодит ажилладаг мэт загвар гаргана.

\textbf{- Usability туршилт хийж дизайнаа сайжруулах}

Гаргасан Prototype хувилбараа сонгож авсан хэрэглэгчдээр ашиглиулж UX дээр ямар алдаа байгаа, хэрэглэгчдийн асуудлыг шийдвэрлэж чадаж байгаа эсэх, вэб компонентүүдийн байрлал, хоорондын логик үйлдэл зөв байгаа эсэхийг тодорхойлж хэрэглэгчдээс санал хүсэлт авах шаардлагатай. Түүний дараагаас дизайнаа дахин сайжруулж, хөгжүүлэлтийн шатанд ороход бэлэн болгох хэрэгтэй.

\subsection{User Interface дизайны шаардлага}

\begin{table}[h]
	\centering
	\caption{User Interface дизайны шаардлага}
	\begin{tabular}{ |p{2cm}|p{13cm}| }
	\hline
	ИДШ 101 &  Шинэлэг дизайнтай байх \\ \hline
	ИДШ 102 &  Гол элементүүд дээр "D85888" hex кодтой өнгийг ашиглах \\ \hline
	ИДШ 103 &  Компонентуудын micro-interaction ойлгомжтой байх \\ \hline
	ИДШ 104 &  Дэвсгэр өнгө дээр цагаан өнгийг түлхүү ашиглах \\ \hline
	ИДШ 105 &  Contrast буюу өнгөний ялгарлыг бага байлгах \\ \hline
	ИДШ 106 &  Веб фонтын хувьд "Rubik - Cyrillic Extented" хувилбарыг ашигласан байх \\ \hline
	ИДШ 107 &  Элемент хооронд white-space буюу сул зайг сайн гаргаж өгөх \\  \hline
\end{tabular}
\end{table}